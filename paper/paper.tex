%% This is file `sample-sigconf.tex',
%% generated with the docstrip utility.
%%
%% The original source files were:
%%
%% samples.dtx  (with options: `sigconf')
%% 
%% IMPORTANT NOTICE:
%% 
%% For the copyright see the source file.
%% 
%% Any modified versions of this file must be renamed
%% with new filenames distinct from sample-sigconf.tex.
%% 
%% For distribution of the original source see the terms
%% for copying and modification in the file samples.dtx.
%% 
%% This generated file may be distributed as long as the
%% original source files, as listed above, are part of the
%% same distribution. (The sources need not necessarily be
%% in the same archive or directory.)
%%
%% The first command in your LaTeX source must be the \documentclass
%% command.
\documentclass[sigconf,review,anonymous]{acmart}

%%
%% \BibTeX command to typeset BibTeX logo in the docs
\AtBeginDocument{%
  \providecommand\BibTeX{{%
    \normalfont B\kern-0.5em{\scshape i\kern-0.25em
      b}\kern-0.8em\TeX}}}

\usepackage{booktabs}
\usepackage{multirow}
\usepackage{tikz}
\usetikzlibrary{matrix,fit,shapes,calc,positioning,shadows,arrows,shapes,backgrounds,decorations.markings,fadings}
\usepackage{listings}
\usepackage[caption=false, font=footnotesize]{subfig}
%%%%%%%%%%%%% code listing
\renewcommand{\ttdefault}{pcr}
\lstset{
  basicstyle=\scriptsize\ttfamily,
  keywordstyle=\scriptsize\ttfamily\bfseries,
  language=Java,             % choose the language of the code
  frame=single,              % adds a frame around the code
  aboveskip=0pt,
  belowskip=0pt,
  breaklines=true,           % sets automatic line breaking
  breakatwhitespace=false,   % sets if automatic breaks should only happen at
  showspaces=false,
  %numbersep=5pt,              % Abstand der Nummern zum Text
  %tabsize=2,                  % Groesse von Tabs
  %extendedchars=true,         %
  %breaklines=true,            % Zeilen werden Umgebrochen
  keywords=[2]{class, incorporateUserFeedback, testPushPop, testPopPush},
}
\usepackage{balance}
\usepackage{wrapfig}
\usepackage{enumitem}
\usepackage{color, colortbl}
\definecolor{Gray}{gray}{0.9}

\newcommand{\tname}{\textsc{Syrius}} %% name of the technique
\newcommand{\ie}{i.e.}
\newcommand{\eg}{e.g.}
\newcommand{\aka}{a.k.a.}
\newcommand{\etal}{and colleagues}
\newcommand{\nids}{NIDS}
\newcommand{\metas}{Metasploit}
\newcommand{\suri}{Suricata}
\newcommand{\numrulessuri}{27.8K}
\newcommand{\percRulesWithContent}{93.5\%}
\newcommand{\numundetected}{\Fix{XX\%}}
\newcommand{\CodeIn}[1]{{\small{\texttt{#1}}}}
\newcommand{\MyComment}[1]{}

%% review
\newcommand{\Fix}[1]{{\textbf{[[}\color{magenta}#1}\textbf{]]}}
\newcommand{\Mar}[1]{{\textbf{[[Marcelo:~}\color{red}#1}\textbf{]]}}
\newcommand{\Luc}[1]{{\textbf{[[Lucas:~}\color{blue}#1}\textbf{]]}}
\newcommand{\Gui}[1]{{\textbf{[[Guilherme:~}\color{green}#1}\textbf{]]}}

\def\denseitems{
   \itemsep1pt plus1pt minus1pt
   \parsep0pt plus0pt
   \parskip0pt\topsep0pt}

%% numbers
\newcommand{\totoptions}{162}
\newcommand{\numproto}{11}
\newcommand{\totoptionsrelevant}{153}



%% Rights management information.  This information is sent to you
%% when you complete the rights form.  These commands have SAMPLE
%% values in them; it is your responsibility as an author to replace
%% the commands and values with those provided to you when you
%% complete the rights form.
\setcopyright{acmcopyright}
\copyrightyear{2018}
\acmYear{2018}
\acmDOI{10.1145/1122445.1122456}

%% These commands are for a PROCEEDINGS abstract or paper.
%% \acmConference[Woodstock '18]{Woodstock '18: ACM Symposium on Neural
%%   Gaze Detection}{June 03--05, 2018}{Woodstock, NY}
%% \acmBooktitle{Woodstock '18: ACM Symposium on Neural Gaze Detection,
%%   June 03--05, 2018, Woodstock, NY}
%% \acmPrice{15.00}
%% \acmISBN{978-1-4503-9999-9/18/06}


%%
%% Submission ID.
%% Use this when submitting an article to a sponsored event. You'll
%% receive a unique submission ID from the organizers
%% of the event, and this ID should be used as the parameter to this command.
%%\acmSubmissionID{123-A56-BU3}

%%
%% The majority of ACM publications use numbered citations and
%% references.  The command \citestyle{authoryear} switches to the
%% "author year" style.
%%
%% If you are preparing content for an event
%% sponsored by ACM SIGGRAPH, you must use the "author year" style of
%% citations and references.
%% Uncommenting
%% the next command will enable that style.
%%\citestyle{acmauthoryear}

%%
%% end of the preamble, start of the body of the document source.
\begin{document}

%%
%% The "title" command has an optional parameter,
%% allowing the author to define a "short title" to be used in page
%% headers.
\title{Synthesis of Intrusion Detection Rules \\ for Increased Network Security}

%%
%% The "author" command and its associated commands are used to define
%% the authors and their affiliations.
%% Of note is the shared affiliation of the first two authors, and the
%% "authornote" and "authornotemark" commands
%% used to denote shared contribution to the research.
\author{Ben Trovato}
\authornote{Both authors contributed equally to this research.}
\email{trovato@corporation.com}
\orcid{1234-5678-9012}
\author{G.K.M. Tobin}
\authornotemark[1]
\email{webmaster@marysville-ohio.com}
\affiliation{%
  \institution{Institute for Clarity in Documentation}
  \streetaddress{P.O. Box 1212}
  \city{Dublin}
  \state{Ohio}
  \postcode{43017-6221}
}

\affiliation{%
  \institution{The Th{\o}rv{\"a}ld Group}
  \streetaddress{1 Th{\o}rv{\"a}ld Circle}
  \city{Hekla}
  \country{Iceland}}
\email{larst@affiliation.org}

\affiliation{%
  \institution{Inria Paris-Rocquencourt}
  \city{Rocquencourt}
  \country{France}
}

\author{Aparna Patel}
\affiliation{%
 \institution{Rajiv Gandhi University}
 \streetaddress{Rono-Hills}
 \city{Doimukh}
 \state{Arunachal Pradesh}
 \country{India}}

\author{Huifen Chan}
\affiliation{%
  \institution{Tsinghua University}
  \streetaddress{30 Shuangqing Rd}
  \city{Haidian Qu}
  \state{Beijing Shi}
  \country{China}}

\author{Charles Palmer}
\affiliation{%
  \institution{Palmer Research Laboratories}
  \streetaddress{8600 Datapoint Drive}
  \city{San Antonio}
  \state{Texas}
  \postcode{78229}}
\email{cpalmer@prl.com}


%%
%% By default, the full list of authors will be used in the page
%% headers. Often, this list is too long, and will overlap
%% other information printed in the page headers. This command allows
%% the author to define a more concise list
%% of authors' names for this purpose.
\renewcommand{\shortauthors}{Trovato and Tobin, et al.}

%%
%% The abstract is a short summary of the work to be presented in the
%% article.
\begin{abstract}
\Fix{elaborate}
\end{abstract}

%%
%% The code below is generated by the tool at http://dl.acm.org/ccs.cfm.
%% Please copy and paste the code instead of the example below.
%%
\begin{CCSXML}
<ccs2012>
 <concept>
  <concept_id>10010520.10010553.10010562</concept_id>
  <concept_desc>Computer systems organization~Embedded systems</concept_desc>
  <concept_significance>500</concept_significance>
 </concept>
 <concept>
  <concept_id>10010520.10010575.10010755</concept_id>
  <concept_desc>Computer systems organization~Redundancy</concept_desc>
  <concept_significance>300</concept_significance>
 </concept>
 <concept>
  <concept_id>10010520.10010553.10010554</concept_id>
  <concept_desc>Computer systems organization~Robotics</concept_desc>
  <concept_significance>100</concept_significance>
 </concept>
 <concept>
  <concept_id>10003033.10003083.10003095</concept_id>
  <concept_desc>Networks~Network reliability</concept_desc>
  <concept_significance>100</concept_significance>
 </concept>
</ccs2012>
\end{CCSXML}

\ccsdesc[500]{Computer systems organization~Embedded systems}
\ccsdesc[300]{Computer systems organization~Redundancy}
\ccsdesc{Computer systems organization~Robotics}
\ccsdesc[100]{Networks~Network reliability}

%%
%% Keywords. The author(s) should pick words that accurately describe
%% the work being presented. Separate the keywords with commas.
\keywords{datasets, neural networks, gaze detection, text tagging}

%% A "teaser" image appears between the author and affiliation
%% information and the body of the document, and typically spans the
%% page.
%% \begin{teaserfigure}
%%   \includegraphics[width=\textwidth]{sampleteaser}
%%   \caption{Seattle Mariners at Spring Training, 2010.}
%%   \Description{Enjoying the baseball game from the third-base
%%   seats. Ichiro Suzuki preparing to bat.}
%%   \label{fig:teaser}
%% \end{teaserfigure}

%%
%% This command processes the author and affiliation and title
%% information and builds the first part of the formatted document.
\maketitle

\section{Introduction}

Network Intrusion Detection Systems (\nids{}) are an important
instrument that system administrators have to secure networks. These
software systems monitor the network traffic for malicious behavior
and act accordingly by blocking messages or alerting humans about
suspicious events. \nids{} are typically placed behind a firewall,
vetting the traffic that the firewall did not block. \nids{} are
frequently adopted today. Various open-source (\eg{}, Snort~\Fix{cite}
and Suricata~\Fix{cite}) and commercial implementations (\eg{},
\Fix{...}  and \Fix{...}) exist. \Fix{...elaborate...}

\sloppy \nids{} can be categorized in two groups: 1)~rule-based and
2)~anomaly-based. A rule-based \nids{} checks if the network traffic
matches a given set of rules. A rule describes a potentially
mallicious message pattern. Figure~\Fix{small example} shows an
example rule of Suricata~\cite{suricata}, a popular open-source
\nids{}, maintained by the Open Information Security Foundation
(OISF)~\cite{oisf}. \Fix{quickly explain example} Rule-based \nids{}
focuses on known attacks whereas anomaly-based \nids{} focuses on
unanticipated issues. The hypothesis of anomaly-based \nids\ is that
attack manifestations produces traffic that deviates from regular
benign traffic that has been already observed by the network and that
deviation is noticeable. Anomaly-based \nids{}, specially those that
use machine learning, have shown useful specially for detecting denial
of service attacks~\Fix{cite cite}. In that scenario, determining the
right threshold on the rate/volume of network accesses that should
trigger an alarm is challenging. Conceptually, these two approaches
are complementary. This paper focuses on rule-based \nids{}, which are
already very popular both in industry and academia. Our analysis
focuses on Suricata, but it applies to any other rule-based \nids{}.

%Despite the great success of rule-based \nids{},

Mantaining \nids{} rules can be a daunting activity. Although the
\nids{} typically provides a set of pre-defined rules, such ruleset is
neither necessary nor sufficient. More precisely:

\begin{enumerate}
\item enabling all possible rules is not practical;
\item rules not in the original rule set may need to be added in a
  given setup.
\end{enumerate}

The first statement is consequential of the inherent physical
limitation of rule-based \nids{}. The cost of checking rules is
proportional to the traffic load and the number of enabled rules. If
system capacity is insufficient to support the load, the system can
potentially miss packets and attacks\Fix{I guess we have some
  reference for this.}. To cope with that issue, system administrators
restrict the attack surface before applying \nids{}---they block
certain protocols on the firewall and configure the \nids{} to use a
selection of rules that apply to the authorized traffic. The second
statement is supported by the observation that \numundetected{} of the
attacks produced by Metasploit~\cite{metasploit}, a popular
open-source penetration testing tool, went undetected by \suri{}. For
example, \Fix{...}. (Section~\ref{sec:suri-metas-coverage} provides
more details.) Unfortunately, manually creating those rules from
scratch can be time-consuming and error-prone. Figure~\Fix{...}  shows
an example rule illustrating the complexity of the Suricata rule
language\footnote{Suricata extends the language used by
  Snort~\cite{snort}.}.  In addition, looking for rules from untrusted
sources (say, the internet) is unreliable. Collectively, these issues
suggest that synthesizing \nids{} rules is an important aid for system
administrators.

\Fix{why is that challenging (worth of research)?}

This paper makes the following contributions. \Fix{...}

\section{\suri{} coverage on the \metas{} dataset}
\label{sec:suri-metas-coverage}

\section{Approach}

\Fix{...}

\section{Evaluation}

\Fix{...}

\section{Threats to Validity}

External Validity. \Fix{elaborate ...how popular are Suricata/Metasploit?}

\section{Related Work}

\Fix{...}

%% The acknowledgments section is defined using the "acks" environment
%% (and NOT an unnumbered section). This ensures the proper
%% identification of the section in the article metadata, and the
%% consistent spelling of the heading.
%% \begin{acks}
%% To Robert, for the bagels and explaining CMYK and color spaces.
%% \end{acks}

%%
%% The next two lines define the bibliography style to be used, and
%% the bibliography file.
\balance \bibliographystyle{ACM-Reference-Format}
\bibliography{references}


\end{document}
\endinput
%%
%% End of file `sample-sigconf.tex'.
